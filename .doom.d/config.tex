% Created 2020-06-26 Fri 11:49
% Intended LaTeX compiler: pdflatex
\documentclass[11pt]{article}
\usepackage{float}
\usepackage{booktabs}
\usepackage{lscape}
\usepackage{hyphenat}
\usepackage{microtype}
\usepackage{tikz}
\usepackage{lmodern}
\usepackage[left=3.0cm, right=3.0cm,top=2.5cm,bottom=3cm]{geometry}
\usepackage{graphicx}
\usepackage{longtable}
\usepackage{float}
\usepackage{wrapfig}
\usepackage{rotating}
\usepackage[normalem]{ulem}
\usepackage{amsmath}
\usepackage{textcomp}
\usepackage{marvosym}
\usepackage{wasysym}
\usepackage{amssymb}
\usepackage{amsmath}
\usepackage[theorems, skins]{tcolorbox}
\usepackage[version=3]{mhchem}
\usepackage[numbers,super,sort&compress]{natbib}
\usepackage{natmove}
\usepackage{url}
\usepackage[cache=false]{minted}
\usepackage{listings}
\usepackage[linktocpage,pdfstartview=FitH,colorlinks,
linkcolor=blue,anchorcolor=blue,
citecolor=blue,filecolor=blue,menucolor=blue,urlcolor=blue]{hyperref}
\usepackage[onehalfspacing]{setspace}
\usepackage{attachfile}
\author{valentin\textsubscript{lechner}}
\date{\today}
\title{Org Mode}
\hypersetup{
 pdfauthor={valentin\textsubscript{lechner}},
 pdftitle={Org Mode},
 pdfkeywords={},
 pdfsubject={},
 pdfcreator={Emacs 26.3 (Org mode 9.4)}, 
 pdflang={Germanb}}
\begin{document}

\maketitle
\tableofcontents

\begin{verbatim}
(org-babel-do-load-languages
 'org-babel-load-languages
 '(
   (sh . t)
   (plantuml . t)
   (python . t)
 )
)
\end{verbatim}
my default directory for org files
\begin{verbatim}
(setq
 org-directory "~/Daten/cloud/tlaloc/org/"
 org-archive-location (concat org-directory ".archive/$s::")
)
\end{verbatim}
This right here tells org to ignore ':ignore' tags, but to include the content
of them which i use for writing my thesis
\begin{verbatim}
(after! org
  (require 'ox-extra)
  (ox-extras-activate '(ignore-headlines))
  (ox-extras-activate '(latex-header-blocks ignore-headlines)))
\end{verbatim}
\section{Look and feel}
\label{sec:org46e6093}
I like those as characters
\begin{verbatim}
(setq
  org-ellipsis " ▼ "
  org-superstar-headline-bullets-list '("#" "■" "◆" "▲" "▶")
  ;; org-superstar-headline-bullets-list '("✡" "⎈" "✽" "✲" "✱" "✻" "✼" "✽" "✾" "✿" "❀" "❁" "❂" "❃" "❄" "❅" "❆" "❇")
  ;;org-superstar-headline-bullets-list '("#" "◉" "○" "✜" "✿""■" "◆" "▲" "▶" )
)
\end{verbatim}
\begin{verbatim}
;; (custom-theme-set-faces
;;  'user
;;  '(variable-pitch ((t (:family "Vollkorn" :height 180 :weight thin))))
;;  '(fixed-pitch ((t ( :family "Fira Code Retina" :height 160)))))
;;
;; (custom-theme-set-faces
;;  'user
;;  '(org-block ((t (:inherit fixed-pitch))))
;;  '(org-code ((t (:inherit (shadow fixed-pitch)))))
;;  '(org-document-info ((t (:foreground "dark violet"))))
;;  '(org-document-info-keyword ((t (:inherit (shadow fixed-pitch)))))
;;  '(org-indent ((t (:inherit (org-hide fixed-pitch)))))
;;  '(org-link ((t (:foreground "royal blue" :underline t))))
;;  '(org-meta-line ((t (:inherit (font-lock-comment-face fixed-pitch)))))
;;  '(org-property-value ((t (:inherit fixed-pitch))) t)
;;  '(org-special-keyword ((t (:inherit (font-lock-comment-face fixed-pitch)))))
;;  '(org-table ((t (:inherit fixed-pitch :foreground "#f1f1f0"))))
;;  '(org-tag ((t (:inherit (shadow fixed-pitch) :weight bold :height 0.8))))
;;  '(org-verbatim ((t (:inherit (shadow fixed-pitch))))))

\end{verbatim}
Hide emphasis markers
\begin{verbatim}
(setq org-hide-emphasis-markers t)
\end{verbatim}
Prettify my lists
Views a • instead of a - (only in lists)
Views a ◦ instead of a + (in lists, that is)
\begin{verbatim}
(font-lock-add-keywords 'org-mode
                        '(("^ *\\([-]\\) "
                           (0 (prog1 () (compose-region (match-beginning 1) (match-end 1) "•"))))))
(font-lock-add-keywords 'org-mode
                        '(("^ *\\([+]\\) "
                           (0 (prog1 () (compose-region (match-beginning 1) (match-end 1) "◦"))))))
\end{verbatim}
Also, I only ever need the last chapters asterisk
\begin{verbatim}
(setq org-hide-leading-stars t)
\end{verbatim}
Syntax highlighting in code
\begin{verbatim}
(setq org-src-fontify-natively t)
\end{verbatim}
Show pretty stuff in org mode
\begin{verbatim}
(setq org-pretty-entities t)
(add-hook 'org-mode-hook 'variable-pitch-mode)
\end{verbatim}
Enable variable pitch mode for changing font
\begin{verbatim}
(add-hook 'org-mode-hook 'variable-pitch-mode)
\end{verbatim}
set up proportional font for org mode
variable tuple gets filled with the first font found and is used
in the block beneath as scaling font
\begin{verbatim}
;;(let* (
;;       (variable-tuple
;;        (cond (
;;               (x-list-fonts "ETBembo") '(:font "ETBembo"))
;;              ((x-list-fonts "Source Sans Pro") '(:font "Source Sans Pro"))
;;              (nil
;;               (warn "Cannot find a Sans Serif Font.  Install Source Sans Pro."))
;;              )
;;        )
;;       (headline `(:inherit default :weight bold :foreground "#5af78e"))
;;       )
;;
;;  (custom-theme-set-faces
;;   'user
;;   `(org-level-8 ((t (
;;                      :inherit default
;;                      :weight bold
;;                      :foreground "#ff6ac1"
;;                      ,@variable-tuple))))
;;   `(org-level-7 ((t (
;;                      :inherit default
;;                      :weight bold
;;                      :foreground "#5af78e"
;;                      ,@variable-tuple))))
;;   `(org-level-6 ((t (
;;                      :inherit default
;;                      :weight bold
;;                      :foreground "#57c7ff"
;;                      ,@variable-tuple))))
;;   `(org-level-5 ((t (
;;                      :inherit default
;;                      :weight bold
;;                      :foreground "#ff5c57"
;;                      ,@variable-tuple))))
;;   `(org-level-4 ((t (
;;                      :inherit default
;;                      :weight bold
;;                      :foreground "#9aeedf"
;;                      ,@variable-tuple
;;                      :height 1.1))))
;;   `(org-level-3 ((t (
;;                      :inherit default
;;                      :weight bold
;;                      :foreground "#f3f99d"
;;                      ,@variable-tuple
;;                      :height 1.25))))
;;   `(org-level-2 ((t (
;;                      :inherit default
;;                      :weight bold
;;                      :foreground "#ff6ac1"
;;                      ,@variable-tuple
;;                      :height 1.5))))
;;   `(org-level-1 ((t (
;;                      :inherit default
;;                      :weight bold
;;                      :foreground "#5af78e"
;;                      ,@variable-tuple
;;                      :height 1.75))))
;;   `(org-document-title ((t (
;;                             :inherit default
;;                             :weight bold
;;                             :foreground "#57c7ff"
;;                             ,@variable-tuple
;;                             :height 2.0
;;                             :underline nil))))))
\end{verbatim}
\begin{verbatim}
(setq
 org-fontify-whole-heading-line t
 org-fontify-done-headline t
 org-fontify-quote-and-verse-blocks t)
\end{verbatim}

I don't like long columns. They are hard to parse - and even harder to navigate
using vim. I tend to do stuff like linebreaks after every
sentence and don't export linebreaks (org), but this is more solid as
it automatically adds a smart linebreak after 80 characters (smart meaning,
don't break my words or my code)
\begin{verbatim}
(add-hook 'org-mode-hook '(lambda () (setq fill-column 80)))
(add-hook 'org-mode-hook 'auto-fill-mode)
\end{verbatim}
\begin{verbatim}
(setq org-enforce-todo-dependencies t)
(setq org-insert-heading-respect-content nil)
(setq org-reverse-note-order nil)
(setq org-deadline-warning-days 7)
(setq org-blank-before-new-entry (quote ((heading . t)
                                         (plain-list-item . nil))))
\end{verbatim}
Smart return does add new list item, … if appropriate
note: if you're on a link, it opens the link
\begin{itemize}
\item \url{http://irreal.org/blog/?p=6131}
\item \url{http://kitchingroup.cheme.cmu.edu/blog/2017/04/09/A-better-return-in-org-mode/}
\end{itemize}


\begin{verbatim}
(require 'org-inlinetask)

(defun scimax/org-return (&optional ignore)
  "Add new list item, heading or table row with RET.
A double return on an empty element deletes it.
Use a prefix arg to get regular RET. "
  (interactive "P")
  (if ignore
      (org-return)
    (cond
     ((eq 'line-break (car (org-element-context)))
      (org-return-indent))

     ;; Open links like usual, unless point is at the end of a line.
     ;; and if at beginning of line, just press enter.
     ((or (and (eq 'link (car (org-element-context))) (not (eolp)))
          (bolp))
      (org-return))

     ;; It doesn't make sense to add headings in inline tasks. Thanks Anders
     ;; Johansson!
     ((org-inlinetask-in-task-p)
      (org-return))

     ;; checkboxes too
     ((org-at-item-checkbox-p)
      (org-insert-todo-heading nil))

     ;; lists end with two blank lines, so we need to make sure we are also not
     ;; at the beginning of a line to avoid a loop where a new entry gets
     ;; created with only one blank line.
     ((org-in-item-p)
      (if (save-excursion (beginning-of-line) (org-element-property :contents-begin (org-element-context)))
          (org-insert-heading)
        (beginning-of-line)
        (delete-region (line-beginning-position) (line-end-position))
        (org-return)))

     ;; org-heading
     ((org-at-heading-p)
      (if (not (string= "" (org-element-property :title (org-element-context))))
          (progn (org-end-of-meta-data)
                 (org-insert-heading-respect-content)
                 (outline-show-entry))
        (beginning-of-line)
        (setf (buffer-substring
               (line-beginning-position) (line-end-position)) "")))

     ;; tables
     ((org-at-table-p)
      (if (-any?
           (lambda (x) (not (string= "" x)))
           (nth
            (- (org-table-current-dline) 1)
            (org-table-to-lisp)))
          (org-return)
        ;; empty row
        (beginning-of-line)
        (setf (buffer-substring
               (line-beginning-position) (line-end-position)) "")
        (org-return)))

     ;; fall-through case
     (t
      (org-return)))))


(define-key org-mode-map (kbd "RET")
  'scimax/org-return)

\end{verbatim}
There's some functions I got from \url{http://doc.norang.ca/org-mode.html}, I just
 modified it to use C-f5 instead of S-f5:
 f5 and C-f5 are bound the functions for narrowing and widening the emacs buffer as defined below.

We now use:

T (tasks) for C-c / t on the current buffer
N (narrow) narrows to this task subtree
U (up) narrows to the immediate parent task subtree without moving
P (project) narrows to the parent project subtree without moving
F (file) narrows to the current file or file of the existing restriction

The agenda keeps widening the org buffer so this gives a convenient way to focus on what we are doing.
\begin{verbatim}

(global-set-key (kbd "<f5>") 'bh/org-todo)
(global-set-key (kbd "C-<f5>") 'bh/widen)

(defun bh/widen ()
  "This here widens a restricted subtree"
  (interactive)
  (if (equal major-mode 'org-agenda-mode)
      (progn
        (org-agenda-remove-restriction-lock)
        (when org-agenda-sticky
          (org-agenda-redo)))
    (widen)))



(defun bh/org-todo (arg)
  "This filters a subtree by todos"
  (interactive "p")
  (if (equal arg 4)
      (save-restriction
        (bh/narrow-to-org-subtree)
        (org-show-todo-tree nil))
    (bh/narrow-to-org-subtree)
    (org-show-todo-tree nil)))

(defun bh/narrow-to-org-subtree ()
  (widen)
  (org-narrow-to-subtree)
  (save-restriction
    (org-agenda-set-restriction-lock)))
\end{verbatim}
\section{Deft}
\label{sec:org88821dc}
\begin{verbatim}
(setq deft-extensions '("org"))
(setq deft-directory "~/Daten/cloud/tlaloc/org")
\end{verbatim}
\section{Capture}
\label{sec:org6d26c2c}
\begin{verbatim}
(require 'org-roam-protocol)
\end{verbatim}
\begin{verbatim}
(setq org-capture-templates `(
    ("p" "Protocol" entry (file+headline ,(concat org-directory "notes.org") "Inbox")
        "* %^{Title}\nSource: %u, %c\n #+BEGIN_QUOTE\n%i\n#+END_QUOTE\n\n\n%?")
    ("L" "Protocol Link" entry (file+headline ,(concat org-directory "notes.org") "Inbox")
        "* %? [[%:link][%:description]] \nCaptured On: %U")
))
;;(setq org-capture-templates
;;      (doct `(("Personal todo" :keys "t"
;;               :icon ("checklist" :set "octicon" :color "green")
;;               :file +org-capture-todo-file
;;               :prepend t
;;               :headline "Inbox"
;;               :type entry
;;               :template ("* TODO %?"
;;                          "%i %a")
;;               )
;;              ("Personal note" :keys "n"
;;               :icon ("sticky-note-o" :set "faicon" :color "green")
;;               :file +org-capture-todo-file
;;               :prepend t
;;               :headline "Inbox"
;;               :type entry
;;               :template ("* %?"
;;                          "%i %a")
;;               )
;;              ("University" :keys "u"
;;               :icon ("graduation-cap" :set "faicon" :color "purple")
;;               :file +org-capture-todo-file
;;                   :headline "University"
;;                   :unit-prompt ,(format "%%^{Unit|%s}" (string-join +org-capture-uni-units "|"))
;;                   :prepend t
;;                   :type entry
;;                   :children (("Test" :keys "t"
;;                               :icon ("timer" :set "material" :color "red")
;;                               :template ("* TODO [#C] %{unit-prompt} %? :uni:tests:"
;;                                          "SCHEDULED: %^{Test date:}T"
;;                                          "%i %a"))
;;                              ("Assignment" :keys "a"
;;                               :icon ("library_books" :set "material" :color "orange")
;;                               :template ("* TODO [#B] %{unit-prompt} %? :uni:assignments:"
;;                                          "DEADLINE: %^{Due date:}T"
;;                                          "%i %a"))
;;                              ("Lecture" :keys "l"
;;                               :icon ("keynote" :set "fileicon" :color "orange")
;;                               :template ("* TODO [#C] %{unit-prompt} %? :uni:lecture:"
;;                                          "%i %a"))
;;                              ("Miscellaneous task" :keys "u"
;;                               :icon ("list" :set "faicon" :color "yellow")
;;                               :template ("* TODO [#D] %{unit-prompt} %? :uni:"
;;                                          "%i %a"))))
;;                  ("Email" :keys "e"
;;                   :icon ("envelope" :set "faicon" :color "blue")
;;                   :file +org-capture-todo-file
;;                   :prepend t
;;                   :headline "Inbox"
;;                   :type entry
;;                   :template ("* TODO %^{type|reply to|contact} %\\3 %? :email:"
;;                              "Send an email %^{urgancy|soon|ASAP|anon|at some point|eventually} to %^{recipiant}"
;;                              "about %^{topic}"
;;                              "%U %i %a"))
;;                  ("Interesting" :keys "i"
;;                   :icon ("eye" :set "faicon" :color "lcyan")
;;                   :file +(or )g-capture-todo-file
;;                   :prepend t
;;                   :headline "Interesting"
;;                   :type entry
;;                   :template ("* [ ] %{desc}%? :%{i-type}:"
;;                              "%i %a")
;;                   :children (("Webpage" :keys "w"
;;                               :icon ("globe" :set "faicon" :color "green")
;;                               :desc "%(org-cliplink-capture) "
;;                               :i-type "read:web"
;;                               )
;;                              ("Article" :keys "a"
;;                               :icon ("file-text" :set "octicon" :color "yellow")
;;                               :desc ""
;;                               :i-type "read:reaserch"
;;                               )
;;                              ("\tRecipie" :keys "r"
;;                               :icon ("spoon" :set "faicon" :color "dorange")
;;                               :file +org-capture-recipies
;;                               :headline "Unsorted"
;;                               :template "%(org-chef-get-recipe-from-url)"
;;                               )
;;                              ("Information" :keys "i"
;;                               :icon ("info-circle" :set "faicon" :color "blue")
;;                               :desc ""
;;                               :i-type "read:info"
;;                               )
;;                              ("Idea" :keys "I"
;;                               :icon ("bubble_chart" :set "material" :color "silver")
;;                               :desc ""
;;                               :i-type "idea"
;;                               )))
;;                  ("Tasks" :keys "k"
;;                   :icon ("inbox" :set "octicon" :color "yellow")
;;                   :file +org-capture-todo-file
;;                   :prepend t
;;                   :headline "Tasks"
;;                   :type entry
;;                   :template ("* TODO %? %^G%{extra}"
;;                              "%i %a")
;;                   :children (("General Task" :keys "k"
;;                               :icon ("inbox" :set "octicon" :color "yellow")
;;                               :extra ""
;;                               )
;;                              ("Task with deadline" :keys "d"
;;                               :icon ("timer" :set "material" :color "orange" :v-adjust -0.1)
;;                               :extra "\nDEADLINE: %^{Deadline:}t"
;;                               )
;;                              ("Scheduled Task" :keys "s"
;;                               :icon ("calendar" :set "octicon" :color "orange")
;;                               :extra "\nSCHEDULED: %^{Start time:}t"
;;                               )
;;                              ))
;;                  ("Project" :keys "p"
;;                   :icon ("repo" :set "octicon" :color "silver")
;;                   :prepend t
;;                   :type entry
;;                   :headline "Inbox"
;;                   :template ("* %{time-or-todo} %?"
;;                              "%i"
;;                              "%a")
;;                   :file ""
;;                   :custom (:time-or-todo "")
;;                   :children (("Project-local todo" :keys "t"
;;                               :icon ("checklist" :set "octicon" :color "green")
;;                               :time-or-todo "TODO"
;;                               :file +org-capture-project-todo-file)
;;                              ("Project-local note" :keys "n"
;;                               :icon ("sticky-note" :set "faicon" :color "yellow")
;;                               :time-or-todo "%U"
;;                               :file +org-capture-project-notes-file)
;;                              ("Project-local changelog" :keys "c"
;;                               :icon ("list" :set "faicon" :color "blue")
;;                               :time-or-todo "%U"
;;                               :heading "Unreleased"
;;                               :file +org-capture-project-changelog-file))
;;                   )
;;                  ("\tCentralised project templates"
;;                   :keys "o"
;;                   :type entry
;;                   :prepend t
;;                   :template ("* %{time-or-todo} %?"
;;                              "%i"
;;                              "%a")
;;                   :children (("Project todo"
;;                               :keys "t"
;;                               :prepend nil
;;                               :time-or-todo "TODO"
;;                               :heading "Tasks"
;;                               :file +org-capture-central-project-todo-file)
;;                              ("Project note"
;;                               :keys "n"
;;                               :time-or-todo "%U"
;;                               :heading "Notes"
;;                               :file +org-capture-central-project-notes-file)
;;                              ("Project changelog"
;;                               :keys "c"
;;                               :time-or-todo "%U"
;;                               :heading "Unreleased"
;;                               :file +org-capture-central-project-changelog-file
;;                               )
;;                              )
;;                   )
;;                  )
;;            )
;;#+END_SRC
;;#+BEGIN_SRC emacs-lisp
;;(setq org-capture-templates
;;      (quote
;;       (("w"
;;         "Default template"
;;         entry
;;         (file+headline (concat org-directory "capture.org") "Notes")
;;         "* %^{Title}\n\n  Source: %u, %c\n\n  %i"
;;         :empty-lines 1)
;;        ;; ... more templates here ...
;;        )))
\end{verbatim}
\begin{verbatim}
(setq org-roam-directory (concat org-directory "roam"))
\end{verbatim}
\begin{verbatim}
(setq org-protocol-default-template-key "w")
\end{verbatim}
\section{Refile}
\label{sec:org5f74f78}
Global keybinding to open my refile-file
\begin{verbatim}
(global-set-key (kbd "C-c o")
  (lambda () (interactive) (find-file (concat org-directory "refile.org"))))
\end{verbatim}
\section{Agenda}
\label{sec:org29cfa35}
Add all files in org dir to agenda
\begin{verbatim}
(setq org-agenda-files (list org-directory))
\end{verbatim}

when all children are done change parent todo entry to done
 see here: \url{https://orgmode.org/org.html\#Breaking-Down-Tasks}
\begin{verbatim}
(defun org-summary-todo (n-done n-not-done)
  "Switch entry to DONE when all subentries are done, to TODO otherwise."
  (let (org-log-done org-log-states)   ; turn off logging
    (org-todo (if (= n-not-done 0) "DONE" "TODO"))))

(add-hook 'org-after-todo-statistics-hook 'org-summary-todo)
(setq org-hierarchical-todo-statistics t)
\end{verbatim}
pretty-print states
\begin{verbatim}
(add-hook 'org-mode-hook
          (lambda ()
            (push '("TODO"  . ?▲) prettify-symbols-alist)
            (push '("DONE"  . ?✓) prettify-symbols-alist)
            (push '("CANCELLED"  . ?✘) prettify-symbols-alist)
            (push '("WAITING"  . ?…) prettify-symbols-alist)
            (push '("SOMEDAY"  . ??) prettify-symbols-alist)))
\end{verbatim}
Change font for done tasks
\begin{verbatim}
(setq org-fontify-done-headline t)
(custom-set-faces
 '(org-done ((t (:foreground "PaleGreen"
                 :weight normal
                 :strike-through t))))
 '(org-headline-done
   ((((class color) (min-colors 16) (background dark))
     (:foreground "LightSalmon" :strike-through t)))))
\end{verbatim}
Use C-c a to open the agenda, f12 to open the agenda as list
\begin{verbatim}
(global-set-key (kbd "C-c a") 'org-agenda)
(global-set-key (kbd "<f12>") 'org-agenda-list)
\end{verbatim}
Sorting by time up, prio down and category up in agenda
Sorting by todo up, state up in todo
Sorting tags by priority downwards
\begin{verbatim}
(setq org-agenda-sorting-strategy
  (quote ((agenda time-up priority-down category-up)
          (todo todo-state-up priority-up)
          (tags priority-down))))
\end{verbatim}
Keywords for todos
\begin{verbatim}
  ;; ! = insert timestamp
  ;; @ = insert note
  ;; / = enter state
  ;; (x) = shortcut (after C-c C-t)
  ;; after the |: close todo
(setq
 org-todo-keywords '(
                     (sequence
                      "DELEGATED(l@/!)"
                      "SOMEDAY(f)"
                      "IDEA(i@/!)"
                      "TODO(t@/!)"
                      "STARTED(s@/!)"
                      "NEXT(n@/!)"
                      "WAITING(w@/!)"
                      "|"
                      "DONE(d@/!)"
                      "CANCELED(c@/!)")
                     )
 )
\end{verbatim}
Colorizing the todo keywords
\begin{verbatim}
(setq  org-todo-keyword-faces
  '(("IDEA" . (
               :foreground "light green"
               :weight bold))
    ("NEXT" . (
               :foreground "orange"
               :weight bold))
    ("TODO" . (
               :foreground "yellow"
               :weight bold))
    ("STARTED" . (
                  :foreground "green"
                  :weight bold))
    ("WAITING" . (
                  :foreground "maroon"
                  :weight bold))
    ("CANCELED" . (
                   :foreground "red"
                   :weight bold))
    ("DELEGATED" . (
                    :foreground "sea green"
                    :weight bold))
    ("SOMEDAY" . (
                  :foreground "seashell"
                  :weight bold))
    )
)
\end{verbatim}
org tags
\begin{verbatim}
(setq
  org-tag-persistent-alist
  '((:startgroup . nil)
    ("HOME" . ?h)
    ("RESEARCH" . ?r)
    ("TEACHING" . ?t)
    ("STUDYING" . ?s)
    (:endgroup . nil)
    (:startgroup . nil)
    ("MGMT" . ?m)
    ("OS" . ?o)
    ("DEV" . ?d)
    ("WWW" . ?w)
    (:endgroup . nil)
    (:startgroup . nil)
    ("EASY" . ?e)
    ("MEDIUM" . ?m)
    ("HARD" . ?a)
    (:endgroup . nil)
    ("URGENT" . ?u)
    ("KEY" . ?k)
    ("BONUS" . ?b)
    ("noexport" . ?x)
    )
)
\end{verbatim}
coloring tags
\begin{verbatim}
(setq
  org-tag-faces
  '(
    ("HOME" . (
               :foreground "aquamarine"
               :weight bold))
    ("RESEARCH" . (
                   :foreground "SeaGreen4"
                   :weight bold))
    ("TEACHING" . (
                   :foreground "SpringGreen1"
                   :weight bold))
    ("STUDYING" . (
                   :foreground "SpringGreen4"
                   :weight bold))
    ("OS" . (
             :foreground "coral4"
             :weight bold))
    ("DEV" . (
              :foreground "tomato1"
              :weight bold))
    ("MGMT" . (
               :foreground "yellow1"
               :weight bold))
    ("WWW" . (
              :foreground "gray0"
              :weight bold))
    ("URGENT" . (
                 :foreground "red"
                 :weight bold))
    ("KEY" . (
              :foreground "red"
              :weight bold))
    ("EASY" . (
               :foreground "SeaGreen1"
               :weight bold))
    ("MEDIUM" . (
                 :foreground "yellow"
                 :weight bold))
    ("HARD" . (
               :foreground "red"
               :weight bold))
    ("BONUS" . (
                :foreground "goldenrod1"
                :weight bold))
    ("noexport" .(
                  :foreground "DarkBlue"
                  :weight bold))
    )
  )
\end{verbatim}
Set recurring tasks to state next
\begin{verbatim}
(setq org-todo-repeat-to-state "NEXT")
\end{verbatim}
Use fast tag and todo selection
\begin{verbatim}
(setq
  org-fast-tag-selection-single-key t
  org-use-fast-todo-selection t
)
\end{verbatim}
\section{Org Ref}
\label{sec:orgcfb2bde}
Setting default files for org ref.
Mine are synced via nextcloud
\begin{verbatim}
(setq
 org-ref-default-bibliography "~/Daten/cloud/tlaloc/org/Papers/references.bib"

 org-ref-pdf-directory "~/Daten/cloud/tlaloc/org/Papers/bibtex-pdfs"

 org-ref-bibliography-notes "~/Daten/cloud/tlaloc/org/Papers/notes.org"
 org-ref-open-pdf-function
 (lambda (fpath)
   (start-process "zathura" "*ivy-bibtex-zathura*" "/usr/bin/zathura" fpath))
)
\end{verbatim}
use footcite as default cite
\begin{verbatim}
(setq org-ref-default-citation-link "footcite")
\end{verbatim}
\section{Export}
\label{sec:org8cdfce3}
Prefer user labels instead of internal labels
\begin{verbatim}
(setq org-latex-prefer-user-labels t)
\end{verbatim}
Use smart quotes
smart quotes means converting hyphens to m-dashes and
straight quotes to curly quotes
\begin{verbatim}
(setq org-export-with-smart-quotes t)
\end{verbatim}
\subsection{Languages}
\label{sec:org1a80c01}
\begin{verbatim}
(setq org-export-default-language "de")
\end{verbatim}
\subsection{\LaTeX{}}
\label{sec:org9fc8717}
Using this latex command ensures your bibliography to be set up as well as your glossaries
\begin{verbatim}
(setq
 org-latex-pdf-process
 '("lualatex -shell-escape -interaction nonstopmode -output-directory %o %f"
   "biber %b"
   "makeglossaries $(echo %b | cut -f 1 -d '.')"
   "lualatex -shell-escape -interaction nonstopmode -output-directory %o %f"
   "lualatex -shell-escape -interaction nonstopmode -output-directory %o %f"))
\end{verbatim}
Setup preview commands
\begin{verbatim}
'(org-preview-latex-process-alist
  (quote
   ((dvipng :programs
            ("lualatex" "dvipng")
            :description "dvi > png"
            :message "you need to install the programs: latex and dvipng."
            :image-input-type "dvi"
            :image-output-type "png"
            :image-size-adjust
            (1.0 . 1.0)
            :latex-compiler
            ("lualatex -output-format dvi -interaction nonstopmode -output-directory %o %f")
            :image-converter
            ("dvipng -fg %F -bg %B -D %D -T tight -o %O %f"))
    (dvisvgm :programs
             ("latex" "dvisvgm")
             :description "dvi > svg"
             :message "you need to install the programs: latex and dvisvgm."
             :use-xcolor t
             :image-input-type "xdv"
             :image-output-type "svg"
             :image-size-adjust
             (1.7 . 1.5)
             :latex-compiler
             ("xelatex -no-pdf -interaction nonstopmode -output-directory %o %f")
             :image-converter
             ("dvisvgm %f -n -b min -c %S -o %O"))
    (imagemagick :programs
                 ("latex" "convert")
                 :description "pdf > png"
                 :message "you need to install the programs: latex and imagemagick."
                 :use-xcolor t
                 :image-input-type "pdf"
                 :image-output-type "png"
                 :image-size-adjust
                 (1.0 . 1.0)
                 :latex-compiler
                 ("xelatex -no-pdf -interaction nonstopmode -output-directory %o %f")
                 :image-converter
                 ("convert -density %D -trim -antialias %f -quality 100 %O")))))
\end{verbatim}
My latex classes
First off, the classic koma-article
\begin{verbatim}
(after! ox-latex
  (add-to-list 'org-latex-classes
               '("koma-article"
                 "\\documentclass[ngerman,12pt]{scrartcl}"
                 ("\\section{%s}" . "\\section*{%s}")
                 ("\\subsection{%s}" . "\\subsection*{%s}")
                 ("\\subsubsection{%s}" . "\\subsubsection*{%s}")
                 ("\\paragraph{%s}" . "\\paragraph*{%s}")
                 ("\\subparagraph{%s}" . "\\subparagraph*{%s}"))))
\end{verbatim}
Secondary, mimosis.
Mimosis is a class for writing books.
\begin{verbatim}
(add-to-list 'org-latex-classes
             '("mimosis"
               "\\documentclass{mimosis}
 [NO-DEFAULT-PACKAGES]
 [PACKAGES]
 [EXTRA]
\\newcommand{\\mboxparagraph}[1]{\\paragraph{#1}\\mbox{}\\\\}
\\newcommand{\\mboxsubparagraph}[1]{\\subparagraph{#1}\\mbox{}\\\\}"
               ("\\chapter{%s}" . "\\chapter*{%s}")
               ("\\section{%s}" . "\\section*{%s}")
               ("\\subsection{%s}" . "\\subsection*{%s}")
               ("\\subsubsection{%s}" . "\\subsubsection*{%s}")
               ("\\mboxparagraph{%s}" . "\\mboxparagraph*{%s}")
               ("\\mboxsubparagraph{%s}" . "\\mboxsubparagraph*{%s}")))
\end{verbatim}
I dont use this one (yet).
The third one's a class for publications
\begin{verbatim}
;; Elsarticle is Elsevier class for publications.
(add-to-list 'org-latex-classes
             '("elsarticle"
               "\\documentclass{elsarticle}
 [NO-DEFAULT-PACKAGES]
 [PACKAGES]
 [EXTRA]"
               ("\\section{%s}" . "\\section*{%s}")
               ("\\subsection{%s}" . "\\subsection*{%s}")
               ("\\subsubsection{%s}" . "\\subsubsection*{%s}")
               ("\\paragraph{%s}" . "\\paragraph*{%s}")
               ("\\subparagraph{%s}" . "\\subparagraph*{%s}")))
\end{verbatim}
This is koma-book (scrbook)
\begin{verbatim}
(add-to-list 'org-latex-classes
             '("koma-book"
               "\\documentclass{scrbook}
 [NO-DEFAULT-PACKAGES]
 [PACKAGES]
 [EXTRA]
\\newcommand{\\mboxparagraph}[1]{\\paragraph{#1}\\mbox{}\\\\}
\\newcommand{\\mboxsubparagraph}[1]{\\subparagraph{#1}\\mbox{}\\\\}"
               ("\\chapter{%s}" . "\\chapter*{%s}")
               ("\\section{%s}" . "\\section*{%s}")
               ("\\subsection{%s}" . "\\subsection*{%s}")
               ("\\subsubsection{%s}" . "\\subsubsection*{%s}")
               ("\\mboxparagraph{%s}" . "\\mboxparagraph*{%s}")
               ("\\mboxsubparagraph{%s}" . "\\mboxsubparagraph*{%s}")))
\end{verbatim}

My default packages for latex
\begin{verbatim}
(setq org-latex-default-packages-alist
      '(
        ("" "float" nil)
        ("" "booktabs" nil)
        ("" "lscape" nil)
        ("" "hyphenat" nil)
        ;; drawing
        ("" "microtype" nil)
        ("" "tikz" nil)
        ;; this is for having good fonts
        ("" "lmodern" nil)
        ;; This makes standard margins
        ("left=3.0cm, right=3.0cm,top=2.5cm,bottom=3cm" "geometry" nil)
        ("" "graphicx" t)
        ("" "longtable" nil)
        ("" "float" nil)
        ("" "wrapfig" nil)      ;makes it possible to wrap text around figures
        ("" "rotating" nil)
        ("normalem" "ulem" t)
        ;; These provide math symbols
        ("" "amsmath" t)
        ("" "textcomp" t)
        ("" "marvosym" t)
        ("" "wasysym" t)
        ("" "amssymb" t)
        ("" "amsmath" t)
        ("theorems, skins" "tcolorbox" t)
        ;; used for marking up chemical formulars
        ("version=3" "mhchem" t)
        ("numbers,super,sort&compress" "natbib" nil)
        ("" "natmove" nil)
        ("" "url" nil)
        ;; this is used for syntax highlighting of code
        ("cache=false" "minted" nil)
        ("" "listings" nil)
        ("linktocpage,pdfstartview=FitH,colorlinks,
linkcolor=blue,anchorcolor=blue,
citecolor=blue,filecolor=blue,menucolor=blue,urlcolor=blue"
         "hyperref" nil)
        ("onehalfspacing" "setspace" nil)
        ;; enables you to embed files in pdfs
        ("" "attachfile" nil)
    ))
\end{verbatim}
\subsubsection{Org Async Export}
\label{sec:org8f4c10d}
For having exports as an async process, which doesnt hang up emacs, you also
need a file like \url{./init-org-async.el}
\begin{verbatim}
;;(setq
;; org-export-in-background t
;; org-export-async-init-file (concat doom-private-dir "init-org-async.el"))
\end{verbatim}

\section{Org Habit}
\label{sec:org41a3e9e}
Still need to get used of how to use this.
For now, I'll just leave that commented out
\begin{verbatim}
(after! org
  (add-to-list 'org-modules 'org-habit t))
\end{verbatim}
\end{document}